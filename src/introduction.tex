\chapter*{Введение}							% Заголовок
\addcontentsline{toc}{chapter}{Введение}	% Добавляем его в оглавление

Во многих задачах машинного обучения графы используются для моделирования связей между объектами. Например, анализ социальных графов и сетей, создание рекомендательных систем, транспортные задачи.

Наиболее важная часть решения подобных задач --- это выбор способа измерения расстояния между вершинами. Для этого используются различные метрики, которые отражают разные свойства графа. Наиболее простой способ определить расстояние --- кратчайший путь --- не всегда дает хорошие результаты, потому что этот метод не учитывает связи, которые длиннее, чем самая короткая, и их количество. Другая распространенная метрика --- resistance distance, как и пропорциональная ей commute time distance, учитывает все возможные пути между вершинами. Однако, в работе \cite{von2014hitting} было показано, что при росте количества вершин в графе данные метрики сходятся к функциям, зависящим от степеней вершин и не отражающим глобальных свойств графа. Были предложены другие способы измерить расстояние между вершинами, большинство из которых представляет собой параметрические семейства и при предельных значениях параметров сходится либо к расстоянию кратчайшего пути, либо к resistance distance. В данной работе изучается поведение этих метрических семейств.

Целью работы является исследование зависимости метрик от параметров для трех типов случайных геометрических графов графов:  $\varepsilon$-графов, графов ближайших соседей и полных графов с гауссовским распределением весов ребер.

Для~достижения поставленной цели необходимо было решить следующие задачи:
\begin{enumerate}
  \item Проанализировать существующую литературу по графовым метрикам  
  \item Разработать критерий сравнения метрик
  \item Разработать модель, позволяющую генерировать графы и вычислять расстояния в терминах метрик   
  \item Исследовать поведение метрик в зависимости от параметров графа и параметров метрики
  \item Вычислить оптимальные в смысле выбранного критерия качества значения параметров метрик для каждого из исследуемых типов графов
\end{enumerate}



\clearpage
\chapter*{Заключение}						% Заголовок
\addcontentsline{toc}{chapter}{Заключение}	% Добавляем его в оглавление

В данной работе было рассмотрено большое число параметрических семейств графовых метрик и исследовано их поведение в зависимости от параметра для четырех типов случайных геометрических графов. Сравнение метрик осуществлялось посредством сравнения каждой из них с евклидовым расстоянием между вершинами графа.

На основании проведенных экспериментов можно сделать вывод, что логарифмическое преобразование позволяет значительно улучшить метрики.

Численные исследования позволили найти значения параметров метрических семейств, которые наиболее интересны для практических приложений, а также было выяснено, что возведение элементов матрицы расстояний в степени, отличные от $1.0$, не позволяют получить лучшее приближение евклидового расстояния.

Таким образом, использование графовых метрик, отличных от кратчайшего пути, позволяет приближать евклидовое расстояние между вершинами с высокой точностью и при этом учитывать различные связи (пути) между вершинами, а значит, расстояния, вычисленные с помощью этих метрик, отражают больше информации о структуре графа, чем расстояние кратчайшего пути.

Исходный код для воспроизведения результатов, описанных в данной работе, доступен по адресу https://github.com/vi-nastya/Bachelor-Graph-Metrics

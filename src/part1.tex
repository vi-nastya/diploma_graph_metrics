\chapter{Постановка задачи} \label{chapt1}

\section{Основные определения} \label{sect1_1}

Пусть $G~=~(V, E)$ --- неориентированный граф с множеством вершин $V$ и множеством ребер $E$, $n$ - число вершин. Матрицу смежности невзвешенного графа будем обозначать $A~=~(a_{ij})$, где $a_{ij}~=~1$, если ребро $(v_i, v_j) \in E$ и $a_{ij}~=~0$ в противном случае. Для взвешенных графов будем хранить в этой матрице веса ребер: $a_{ij}~=~w(v_i, v_j)$.

\begin{definition}
\emph{Метрикой на множестве} $X$ называется функция $d: X^2\rightarrow \mathbb{R}$ такая, что для любых $x,\ y,\ z \in X$ выполнены следующие утверждения:
\begin{enumerate}
\item $d(x,y) = 0$\ \ \  тогда и только тогда, когда $x=y$
\item $d(x,y) + d(x,z) - d(y,z) \ge 0$\ \ \  (неравенство треугольника)
\end{enumerate}
\end{definition}

Из этого определения следует, что для любых $x,\ y \in X$:

\begin{enumerate}
\item $d(x,y) = d(y,x)$\ \ \ (симметричность)
\item $d(x,y) \ge 0$\ \ \  (неотрицательность)
\end{enumerate}


Рассмотрим другой класс функций - меры близости, которые широко применяются в теории графов и сетей, исследовании марковских процессов и анализе статистических моделей. На практике метрики часто получают из функций близости. Приведем ряд теорем, показывающих связь между этими классами функций.

\begin{definition}
Пусть $X$ --- непустое множество и $\Sigma \in \mathbb{R}$. Функция $\sigma: X^2 \rightarrow \mathbb{R}$ называется $\Sigma$\emph{-proximity} на $A$, если для любых $x,\ y,\ z \in X$ выполняются следующие условия:
\begin{enumerate}
\item $\sum\limits_{t \in X} \sigma (x,t) = \Sigma$
\item $\sigma(x,y) + \sigma(x,z) - \sigma(y,z) \le \sigma(x,x)$,\ \  где при $z = y$ и $x \ne y$ неравенство строгое.
\end{enumerate}
\end{definition}

В работе \cite{chebotarev2013studying} было доказано, что между метриками и $\Sigma$-proximities на множестве $X$ существует взаимно однозначное соответствие.

\begin{definition}
Пусть $G$ - мультиграф с набором вершин $V$. Функция $d: V*V \rightarrow \mathbb{R}$ называется \emph{graph-geodetic (bottleneck addictive, cutpoint addictive)}, если $d(i,j)+d(j,k) = d(i,k)$ выполнено тогда и только тогда, когда в графе $G$ любой путь, соединяющий вершины $i$ и $k$, проходит через вершину $j$.
\end{definition}

\begin{definition}
Говорят, что матрица $S=(s_{ij}) \in \mathbb{R}^{n\times n}$ задает \emph{транзитивную меру} $s(i,j) = s_{ij}$ на вершинах $i,j \in V$ графа $G$, если ее элементы удовлетворяют транзитивному неравенству $$s_{ij}s_{jk} \le s_{ik}s_{jj}.$$

Это неравенство является аналогом неравенства треугольника для мер близости.
\end{definition}
\textbf{Теорема}
Пусть $S=(s_{ij}) \in \mathbb{R}^{n\times n}$ задает транзитивную меру на графе $G$ и все недиагональные элементы этой матрицы положительны. Тогда матрица $D = (d_{ij})_{n\times n}$, определенная как
$$D~=~(h \textbf{1}^\intercal + \textbf{1} h^\intercal - H - H ^\intercal) /2,$$ 
где $H$ получается поэлементным логарифмированием матрицы $S$, является матрицей расстояний на $V(G)$. Более того, это расстояние будет cutpoint addictive.

Доказательство этой теоремы можно найти в \cite{chebotarev2005duality}.

В данной работе расстояние между вершинами в графе задается матрицей расстояний $D~=~(d_{ij})$, которую получают из определенным образом заданных мер близости $H~=~(h_{ij})$ с помощью преобразования
$$D~=~(h \textbf{1}^\intercal + \textbf{1} h^\intercal - H - H ^\intercal) /2, $$
где $h$ --- вектор-диагональ матрицы $H$.

В некоторых случаях вместо матрицы $H$ можно использовать матрицу $H_0$, состоящую из логарифмов элементов матрицы $H$.


%\newpage
%============================================================================================================================
\clearpage

\section{Задача} \label{sect1_2}

Пусть $G$ --- случайный геометрический граф. Требуется исследовать близость параметрических семейств графовых метрик на этом графе к евклидовому расстоянию между вершинами графа и найти оптимальные параметры, при которых метрики наилучшим образом приближают это расстояние.

Также требуется сравнить поведение логарифмических и нелогарифмических метрик.

В данной работе рассматривались четыре класса случайных геометрических графов: $\varepsilon$-графы, два типа графов ближайших соседей, графы с гауссовским распределением весов ребер. Параметр ($\varepsilon$ или количество соседей) выбирался таким образом, чтобы граф оказался связным с высокой вероятностью. Это делалось потому, что наибольший интерес для машинного обучения представляют именно связные графы.

Для полученных графов вычислялись различные метрики, затем они сравнивались между собой. Для сравнения использовались коэффициент корреляции с евклидовым расстоянием и матричные нормы для матрицы $D_{euclid} - D_{metrics}$.




\clearpage

\section{Исследуемые метрики} \label{sect1_3}
\begin{itemize}
\item[1.] \textbf{Walk distance}

Это параметрическое семейство строится с использованием меры близости 
\begin{equation}
H = (I - tA)^{-1},
\end{equation}
где параметр $0 < t < \rho ^{-1}$, $\rho$ --- спектральный радиус матрицы A. При предельных значениях параметра метрика сходится к shortest path distance и long walk distance. 

\item[2.] \textbf{Logarithmic walk distance}

Мера $H_0$ получается поэлементным логарифмированием матрицы $H$, определяющей Walk distance.

\item[3.] \textbf{e-walk distance}

Является модификацией Walk distance для взвешенных графов 

Веса ребер рассчитываются по следующей формуле: \begin{equation}
w_{ij} = \frac{a_{ij}}{\rho} e^{-\frac{1}{\alpha a_{ij}}},
\end{equation} где $a_{ij}$ - элемент матрицы смежности $A$, $\rho$ - спектральный радиус $A$, $\alpha > 0$ - параметр метрики

\item[4.] \textbf{Forest distance}

Данное семейство подробно описано в \cite{chebotarev2005duality}.

\emph{Rooted tree} --- связный ациклический граф, одна вершина в котором отмечена как корень. \emph{Rooted forest} --- граф, все связные компоненты которого являются rooted trees.

Рассмотрим взвешенный граф $G$. Обозначим за $w(G)$ произведение весов его ребер. Для графа без ребер $w(G) = 1$. Если $S$ --- набор графов, то $w(S) = \sum\limits_{G \in S} w(G)$.  В случае, когда $S$ --- пустое множество, $w(S) = 0$. Если множество $S$ состоит из невзвешенных графов, то $w(S) = |S|$.

Введем следующие обозначения: 

\begin{enumerate}
\item $F = F(G)$ - множество spanning rooted forests графа $G$; 
\item $F_{i,j} = F_{i,j}(G)$- множество таких spanning rooted forests, что вершина $i$ принадлежит дереву с корнем $j$; 
\item $F_{i,j}^{(p)} = F_{i,j}^{(p)}(G)$ - подмножество таких spanning rooted forests множества $F_{i,j}$, которые содержат ровно $p$ ребер.
\end{enumerate}

Пусть 
$$f = w(F),\ \  f_{i,j} = w(F_{i,j}),\ \  f_{i,j}^{(p)} = w(F_{i,j}^{(p)}),$$ 
где $i,j \in V(G)$ и $0 \le p < n$.

Теперь рассмотрим матрицу $Q = (I + L)^{-1}$.

Согласно \emph{Matrix forest theorem}, такая матрица существует для любого взвешенного мультиграфа и ее элементы равны $q_{i,j} = f_{i,j}/f,\ \ i,\ j = 1,\ 2\ldots n$. Матрицу $Q$ можно рассматривать как меру близости. 

Добавим зависимость от параметра:

\begin{equation}
H = (I + tL) ^{-1}, 
\end{equation} 
где параметр $t > 0$, а $L$ --- лапласиан графа.

При $t \to \inf$ данная метрика сходится к resistance distance. Доказательство этого факта, а также интерпретацию метрики можно найти в \cite{chebotarev2012walk}.

\item[5.] \textbf{Logarithmic forest distance}

$H$ получена поэлементным логарифмированием матрицы близости для forest distance

\item[6.] \textbf{Communicability distance}

Данное семейство и его свойства подробно описаны в \cite{estrada2012communicability}.

Communicability между вершинами $p$ и $q$ в графе $G$ - это взвешенная сумма всех блужданий, которые начинаются в $p$ и заканчиваются в $q$, при этом чем короче блуждание, тем больше его вес. Если $A$ - матрица смежности графа, то Communicability между вершинами $p$ и $q$ - это соответствующий элемент матрицы $e^{A}$. 

Данное определение имеет простую физическую интерпретацию. Рассмотрим граф как систему из шариков массой $m$, соединенных пружинами с константой $m \omega ^2$. Затем вся эта система погружается в жидкость с температурой $T$. Под воздействием температуры шарики начинают осциллировать.

Гамильтониан системы имеет следующий вид:

$H = \Sigma _i (\frac{p_i^2}{2m} + (K-k_i)\frac{m\omega ^2 x_i^2}{2}) + \frac{m \omega ^2}{2} \Sigma _{i,j : i<j } A_{ij} (x_i-x_j)^2 $

**See Estrada p2**

\begin{equation}
H = e^{tA},
\end{equation}
 параметр $t > 0$

\item[7.] \textbf{Logarithmic communicability distance}

$H$ получена поэлементным логарифмированием матрицы близости для communicability distance

\item[8.] \textbf{Free energy distance}

Это семейство метрик, зависящее от параметра $\beta$, было рассмотрено в работе \cite{kivimaki2014developments}. Расстояние вычисляется следующим образом:

$P^{ref} = D^{-1} A$, $D = diag(Ae)$

$W = P^{ref} ** e^{-\beta C}$, где $C$ - матрица кратчайших расстояний между вершинами графа $G$

$Z = (I-W)^{-1}$

$Z^h = Z * D_h^{-1}$, $D_h = diag(Z)$

$\Phi = -\frac{1}{\beta} \log Z^h$

\begin{equation}
D^{FE} = (\Phi + \Phi ^T)/ 2
\end{equation}

Данное расстояние стремится к расстоянию кратчайшего пути при $\beta \rightarrow \infty$ и к commute time при $\beta \rightarrow 0^+$

\item[9.] \textbf{Shortest path distance}

Кратчайшим путем между двумя вершинами графа называют такой путь между этими вершинами, что сумма весов ребер, из которых он состоит, минимальна.

Существует несколько способов вычисления кратчайшего пути, в данной работе используется алгоритм Флойда - Уоршелла \cite{floyd1962algorithm}.

\item[10.] \textbf{Resistance distance}

Резисторное расстояние между двумя вершинами эквивалентно напряжению между соответствующими точками в электрической цепи, полученной из графа $G$ заменой ребер на резисторы, сопротивление которых совпадает с весом ребер. 

\begin{equation}
D = (L + J)^{-1},
\end{equation}
 
где $L$ - лапласовская матрица, $J$ - матрица, все элементы которой равны $\frac {1}{n}$, гдк $n$ - число вершин

\item[11.] \textbf{Avrachenkov distance}

Данное семейство мер близости было предложено в \cite{avrachenkov2012generalized}. Оно возникло при исследовании способов решения задачи классификации с частичным привлечением учителя (semi-supervised classification), которые основаны на использовании графов. 

\begin{equation}
H = (1 - a)(I - aD^{-\sigma}AD^{\sigma-1})^{-1},
\end{equation}
где $a = \frac {2}{2+\mu}$,  $\mu$ - параметр регуляризации, который позволяет регулировать баланс между точностью классификации и гладкостью классифицирующей функции. Параметр 
$\sigma$ позволяет использовать общую формулу для трех методов классификации с частичным привлечением учителя. При $\sigma = 1$ получаем метод, основанный на использовании стандартного лапласиана графа, $\sigma = 0.5$ - нормированного лапласиана,  случай $\sigma = 0$ соответствует PageRank.

$D$ - матрица степеней вершин. В случае взвешенных графов вычисляется как сумма весов ребер, инцидентных данной вершине.


\item[12.] \textbf{Logarithmic Avrachenkov distance}

Данная мера близости вычисляется с помощью поэлементного логарифмирования элементов матрицы $H$ для метрики Авранченкова. 

\end{itemize}


\clearpage

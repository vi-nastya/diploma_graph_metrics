\chapter{Постановка задачи} \label{chapt1}

\section{Основные определения} \label{sect1_1}

Пусть $G~=~(V, E)$ --- неориентированный граф с множеством вершин $V$ и множеством ребер $E$, $n$ - число вершин. Матрицу смежности невзвешенного графа будем обозначать $A~=~(a_{ij})$, где $a_{ij}~=~1$, если ребро $(v_i, v_j) \in E$ и $a_{ij}~=~0$ в противном случае. Для взвешенных графов будем хранить в этой матрице веса ребер: $a_{ij}~=~w(v_i, v_j)$. Обозначим $D$ матрицу степеней вершин графа $G$.

Также в работе используются понятия спектрального радиуса матрицы: $\rho(A) = \text{max}_i |\lambda_i(A)|$ и лапласиана графа: $L = D - A$

\begin{definition}
\emph{Метрикой на множестве} $X$ называется функция $d: X^2\rightarrow \mathbb{R}$ такая, что для любых $x,\ y,\ z \in X$ выполнены следующие утверждения:
\begin{enumerate}
\item $d(x,y) = 0$\ \ \  тогда и только тогда, когда $x=y$
\item $d(x,y) + d(x,z) - d(y,z) \ge 0$\ \ \  (неравенство треугольника)
\end{enumerate}
\end{definition}

Из этого определения следует, что для любых $x,\ y \in X$:

\begin{enumerate}
\item $d(x,y) = d(y,x)$\ \ \ (симметричность)
\item $d(x,y) \ge 0$\ \ \  (неотрицательность)
\end{enumerate}


На практике графовые метрики часто получают из функций близости. Они широко применяются в теории графов и сетей, исследовании марковских процессов и анализе статистических моделей. В данной работе рассматриваются два класса функций близости: $\Sigma$\emph{-близости} и \emph{передаточные меры}. Приведем определения этих классов и ряд теорем, показывающих связь между ними и метриками.

\begin{definition}
Пусть $X$ --- непустое множество и $\Sigma \in \mathbb{R}$. Функция $\sigma: X^2 \rightarrow \mathbb{R}$ называется $\Sigma$\emph{-близостью} на $A$, если для любых $x,\ y,\ z \in X$ выполняются следующие условия:
\begin{enumerate}
\item $\sum\limits_{t \in X} \sigma (x,t) = \Sigma$
\item $\sigma(x,y) + \sigma(x,z) - \sigma(y,z) \le \sigma(x,x)$,\ \  где при $z = y$ и $x \ne y$ неравенство строгое.
\end{enumerate}
\end{definition}

В работе \cite{chebotarev2013studying} было доказано, что между метриками и $\Sigma$-proximities на множестве $X$ существует взаимно однозначное соответствие.

\begin{definition}
Пусть $G$ - мультиграф с набором вершин $V$. Функция $d: V*V \rightarrow \mathbb{R}$ называется \emph{граф-геодезической (graph-geodetic), или разрезно-аддитивной (cutpoint addictive)}, если $d(i,j)+d(j,k) = d(i,k)$ выполнено тогда и только тогда, когда в графе $G$ путь, соединяющий вершины $i$ и $k$, проходит через вершину $j$.
\end{definition}

\begin{definition}
Говорят, что матрица $S=(s_{ij}) \in \mathbb{R}^{n\times n}$ задает \emph{передаточную меру} $s(i,j) = s_{ij}$ на вершинах $i,j \in V$ графа $G$, если ее элементы удовлетворяют передаточному неравенству $$s_{ij}s_{jk} \le s_{ik}s_{jj}.$$

Это неравенство является аналогом неравенства треугольника для мер близости.
\end{definition}

\textbf{Теорема}
Пусть $S=(s_{ij}) \in \mathbb{R}^{n\times n}$ задает транзитивную меру на графе $G$ и все недиагональные элементы этой матрицы положительны. Тогда матрица $D = (d_{ij})_{n\times n}$, определенная как
$$D~=~(h \textbf{1}^\intercal + \textbf{1} h^\intercal - H - H ^\intercal) /2,$$ 
где $H$ получается поэлементным логарифмированием матрицы $S$, является матрицей расстояний на $V(G)$. Более того, это расстояние будет cutpoint addictive.

Доказательство этой теоремы можно найти в \cite{chebotarev2005duality}.


В данной работе расстояние между вершинами в графе задается матрицей расстояний $D~=~(d_{ij})$, которую получают из определенным образом заданных мер близости $H~=~(h_{ij})$ с помощью преобразования
$$D~=~(h \textbf{1}^\intercal + \textbf{1} h^\intercal - H - H ^\intercal) /2, $$
где $h$ --- вектор-диагональ матрицы $H$.

В некоторых случаях вместо матрицы $H$ можно использовать матрицу $H_0$, состоящую из логарифмов элементов матрицы $H$.


%\newpage
%============================================================================================================================
\clearpage

\section{Задача} \label{sect1_2}

Пусть $G$ --- случайный геометрический граф. В данной работе рассматриваются четыре класса графов: $\varepsilon$-графы, два типа графов ближайших соседей, графы с гауссовским распределением весов ребер. Требуется исследовать близость параметрических семейств графовых метрик на этом графе к евклидовому расстоянию между вершинами графа и найти оптимальные параметры метрик, при которых метрики наилучшим образом приближают это расстояние. Для этого необходимо выбрать критерий сравнения метрик с евклидовым расстоянием.

Также требуется сравнить поведение логарифмических и нелогарифмических метрик.

Проверяется гипотеза о том, что если перед сравнением возвести все элементы матрицы $D$ в некоторую степень из интервала $(0,1)$, то качество приближения евклидового расстояния может улучшиться. Для каждой метрики требуется найти такую степень.

\clearpage

\section{Исследуемые метрики} \label{sect1_3}

В данной работе рассматриваются следующие параметрические семейства графовых метрик:

\begin{itemize}
\item[1.] \textbf{Маршрутное расстояние (Walk distance)}

Это параметрическое семейство строится с использованием меры близости 
\begin{equation}
H = (I - tA)^{-1},
\end{equation}
где параметр $0 < t < \rho ^{-1}$, $\rho$ --- спектральный радиус матрицы A. При предельных значениях параметра метрика сходится к shortest path distance и long walk distance. Данное семейство задает $\Sigma$-близость, доказательство этого факта в работе \cite{chebotarev2012walk}. Интерпретацию метрики можно найти в \cite{chebotarev2012walk}

\item[2.] \textbf{Логарифмическое маршрутное расстояние (Logarithmic walk distance)}

Мера $H_0$ получается поэлементным логарифмированием матрицы $H$, определяющей Walk distance. Эта матрица задает передаточную меру, доказательство можно найти в работе \cite{chebotarev2011bottleneck}.

\item[3.] \textbf{e-walk distance}

Является модификацией Walk distance для взвешенных графов 

Веса ребер рассчитываются по следующей формуле: \begin{equation}
w_{ij} = \frac{a_{ij}}{\rho} e^{-\frac{1}{\alpha a_{ij}}},
\end{equation} где $a_{ij}$ - элемент матрицы смежности $A$, $\rho$ - спектральный радиус $A$, $\alpha > 0$ - параметр метрики.

Свойства данного семейства и доказательство того, что оно является $\Sigma$-близостью, можно найти в работе \cite{chebotarev2012walk}.

\item[4.] \textbf{Лесное растоние (Forest distance)}

\emph{Корневое дерево (rooted tree)} --- связный ациклический граф, одна вершина в котором отмечена как корень. \emph{Корневой лес (rooted forest)} --- граф, все связные компоненты которого являются rooted trees.

Рассмотрим взвешенный граф $G$. Обозначим за $w(G)$ произведение весов его ребер. Для графа без ребер $w(G) = 1$. Если $S$ --- набор графов, то $w(S) = \sum\limits_{G \in S} w(G)$.  В случае, когда $S$ --- пустое множество, $w(S) = 0$. Если множество $S$ состоит из невзвешенных графов, то $w(S) = |S|$.

Введем следующие обозначения: 

\begin{enumerate}
\item $F = F(G)$ - множество остовных корневых лесов (spanning rooted forests) графа $G$; 
\item $F_{i,j} = F_{i,j}(G)$- множество таких остовных корневых лесов, что вершина $i$ принадлежит дереву с корнем $j$; 
\item $F_{i,j}^{(p)} = F_{i,j}^{(p)}(G)$ - подмножество таких остовных корневых лесов множества $F_{i,j}$, которые содержат ровно $p$ ребер.
\end{enumerate}

Пусть 
$$f = w(F),\ \  f_{i,j} = w(F_{i,j}),\ \  f_{i,j}^{(p)} = w(F_{i,j}^{(p)}),$$ 
где $i,j \in V(G)$ и $0 \le p < n$.

Теперь рассмотрим матрицу $Q = (I + L)^{-1}$.

Согласно Matrix forest theorem, такая матрица существует для любого взвешенного мультиграфа и ее элементы равны $q_{i,j} = f_{i,j}/f,\ \ i,\ j = 1,\ 2\ldots n$. Матрицу $Q$ можно рассматривать как меру близости. 

Добавим зависимость от параметра:

\begin{equation}
H = (I + tL) ^{-1}, 
\end{equation} 
где параметр $t > 0$, а $L$ --- лапласиан графа.

При $t \rightarrow \infty$ данная метрика сходится к resistance distance. Данное семейство задает $\Sigma$-близость и описано в \cite{chebotarev2011class}.

\item[5.] \textbf{Логарифмическое лесное расстояние (Logarithmic forest distance)}

$H$ получена поэлементным логарифмированием матрицы близости для forest distance. Эта матрица задает транзитивную меру, доказательство этого факта и свойства метрики можно найти в работах \cite{chebotarev2011bottleneck}, \cite{chebotarev2002forest} и \cite{chebotarev2011class}.

\item[6.] \textbf{Communicability distance}

Communicability между вершинами $p$ и $q$ в графе $G$ - это взвешенная сумма всех блужданий, которые начинаются в $p$ и заканчиваются в $q$, при этом чем короче блуждание, тем больше его вес. Если $A$ - матрица смежности графа, то Communicability между вершинами $p$ и $q$ - это соответствующий элемент матрицы $e^{A}$. 

Данное определение имеет простую физическую интерпретацию. Рассмотрим граф как систему из шариков массой $m$, соединенных пружинами с константой $m \omega ^2$. Затем вся эта система погружается в жидкость с температурой $T$. Под воздействием температуры шарики начинают осциллировать.

Гамильтониан системы имеет следующий вид:

$$H = \sum\limits_{i} \left(\frac{p_i^2}{2m} + (K-k_i)\frac{m\omega ^2 x_i^2}{2}\right) + \frac{m \omega ^2}{2} \sum\limits_{i,j : i<j} A_{ij} (x_i-x_j)^2, $$

где $k_i$ - степень вершины $i$, $K \ge \text{max}_i k_i$, $x_i$ - координата $i$-го шарика, характеризующая его отклонение от положения равновесия $x_i =0$. Тогда в предположении, что система подчиняется законам квантовой механики, элемент $G_{pq}$ - это термальная функция Грина осциллирующей системы когда обратная температура равна нулю. Следовательно, $G_{pp}$ показывает, какая часть возбуждения узла $p$ передается в систему до того, как оно возвращается обратно и угасает, а элемент $G_{pq}$ показывает, какая часть этого возбуждения передается от вершины $p$ к вершине $q$. 


Функция близости, соответствующая данному расстоянию имеет вид:
\begin{equation}
H = e^{tA},
\end{equation}
 параметр $t > 0$
 
Данное семейство задает $\Sigma$-близость. Его свойства описаны в работе \cite{estrada2012communicability}.

\item[7.] \textbf{Logarithmic communicability distance}

$H$ получена поэлементным логарифмированием матрицы близости для communicability distance. Данное семейство задает транзитивную меру.

\item[8.] \textbf{Расстояние свободных энергий (Free energy distance)}

Это семейство метрик, зависящее от параметра $\beta$, было рассмотрено в работе \cite{kivimaki2014developments}. Физический смысл параметра - температура. Данное расстояние вычисляется следующим образом:

$P^{ref} = D^{-1}A$, $D = \text{diag}(Ae)$, то есть $P^{ref}$ - матрица commute time расстояний между вершинами графа.

$W = P^{ref} \circ e^{-\beta C}$, где $\circ$ означает поэлементное умножение, а элементы матрицы $C$ $c_{ij} = 1 / a_{ij}.$

$Z = (I-W)^{-1},$

$Z^h = ZD_h^{-1}$, $D_h = \text{diag}(Z),$

$\Phi = -\beta^{-1} \log Z^h$ - матрица свободных энергий, логарифмирование поэлементное.

И, наконец:
\begin{equation}
D^{FE} = (\Phi + \Phi ^T)/ 2
\end{equation}

Данное расстояние стремится к расстоянию кратчайшего пути при $\beta \rightarrow \infty$ и к commute time при $\beta \rightarrow 0^+$.

\item[9.] \textbf{Кратчайший путь (Shortest path distance)}

Кратчайшим путем между двумя вершинами графа называют такой путь между этими вершинами, что сумма длин ребер (величин, обратных весам), из которых он состоит, минимальна.

Существует несколько способов вычисления кратчайшего пути, в данной работе используется алгоритм Флойда - Уоршелла \cite{floyd1962algorithm}.

\item[10.] \textbf{Резисторное расстояние (Resistance distance)}

Резисторное расстояние между двумя вершинами эквивалентно эффективному сопротивлению между соответствующими точками в электрической цепи, полученной из графа $G$ заменой ребер на резисторы, сопротивление которых совпадает с весом ребер. 

\begin{equation}
H = (L + J)^{-1},
\end{equation}
 
где $L$ - лапласовская матрица, $J$ - матрица, все элементы которой равны $\frac {1}{n}$, гдк $n$ - число вершин. Данное семейство задает $\Sigma$-близость.

\item[11.] \textbf{Расстояние Авраченкова (Avrachenkov distance)}

Данное семейство мер близости было предложено в \cite{avrachenkov2012generalized}. Оно возникло при исследовании способов решения задачи классификации с частичным привлечением учителя (semi-supervised classification), которые основаны на использовании графов. В данной работе оно впервые рассматривается как функция близости.

\begin{equation}
H = (1 - a)(I - aD^{-\sigma}AD^{\sigma-1})^{-1},
\end{equation}
где $a = 2/(2+\mu)$,  $\mu$ - параметр регуляризации, который позволяет регулировать баланс между точностью классификации и гладкостью классифицирующей функции. Параметр 
$\sigma$ позволяет использовать общую формулу для трех методов классификации с частичным привлечением учителя. При $\sigma = 1$ получаем метод, основанный на использовании стандартного лапласиана графа, $\sigma = 0.5$ - нормированного лапласиана,  случай $\sigma = 0$ соответствует PageRank.

$D$ - матрица степеней вершин. В случае взвешенных графов вычисляется как сумма весов ребер, инцидентных данной вершине.


\item[12.] \textbf{Логарифмическое расстояние Авраченкова (Logarithmic Avrachenkov distance)}

Данная мера близости вычисляется с помощью поэлементного логарифмирования элементов матрицы $H$ для метрики Авраченкова. 

\end{itemize}


\clearpage

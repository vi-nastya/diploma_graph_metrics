\chapter*{Выводы}						% Заголовок
\addcontentsline{toc}{chapter}{Заключение}	% Добавляем его в оглавление

Анализируя результаты экспериментов, можно сделать следующие выводы:

Во-первых, логарифмическое преобразование метрики позволяет значительно улучшить качество приближения евклидового расстояния. Для всех рассмотренных типов графов по всем четырем критериям логарифмические метрики показывают лучшие результаты, чем метрики без логарифма. Данное наблюдение является очень значимым, поскольку в настоящее время логарифмические метрики применяются очень редко.

Во-вторых, можно заметить, что качественно поведение графиков зависимости коэффициентов корреляции и векторных норм от параметра метрики схоже. Максимумы на графиках незначительно меняют свое положение при изменении параметров графа (размерность, число гауссиан в смеси, положение их центров и дисперсии, параметр $\varepsilon$ или $k$) до тех пор, пока граф не начинает распадаться на кластеры, после чего значения коэффициентов корреляции в максимумах начинает уменьшаться.

В-третьих, следует отметить метрику Авраченкова. До данной работы эта функция никогда не рассматривалась в качестве метрики, и эксперименты показали, что ее применени на практике имеет смысл: и сама метрика, и ее логарифм позволяют с высокой точностью восстановить евклидовое расстояние между вершинами исходного графа. Заметим, что в данной работе представлены результаты только для $\sigma=1.0$, потому что для других значений данного параметра зависимость от  параметра $a$ аналогичная. Это связано с близостью степеней вершин в исследуемых графах.


В-четвертых, хотя логарифмические преобразования дают очень хорошие результаты для всех типов графов, в случае гауссовских взвешенных графов расстояние кратчайшего пути позволяет восстановить евклидово расстояние с точностью до константы. Это связано с особенностями определения весов ребер (в случае невзвешенных графов по ребрам можно только понять, меньше ли расстояние между вершинами, чем заданный параметр $\varepsilon$ и попадает ли вершина в число $k$ ближайших соседей другой; в то время как для взвешенных графов веса содержат информацию непосредственно о евклидовом расстоянии между вершинами).


В результате экспериментов для поиска оптимальных степеней выяснилось, что возведение элементов матрицы $D$ в степени, отличные от $1.0$, не улучшают качества приближения евклидового расстояния.

\chapter{Результаты} \label{chapt3}


\section{Незвешенные графы} \label{sect3_1}

Результаты экспериментов представлены на графиках.
Во всех случаях по оси $x$ отложены значения параметра семейства. Для удобства все параметры были отнормированы на отрезок $[0,1]$ с помощью дробно-линейного преобразования.

**Картинки ** 

Можем видеть, что логарифмические метрики позволяют более точно восстановить евклидово расстояние, чем метрики без логарифма.

Значения параметров, при которых метрики лучше всего приближают евклидово расстояние для каждого типа графов, приведены в таблице:

*****

**про оптимальные степени**


%\newpage
%============================================================================================================================


\section{Взвешенные графы} \label{sect3_2}

На графиках представлены результаты сравнения метрик с евклидовым расстоянием для взвешенных Гауссовских графов.

******

Видно, что в данном случае расстояние кратчайшего пути позволяет восстановить евклидово расстояние с точностью до константы. Это связано с особенностями определения весов ребер.

Наилучшие значения параметров для остальных метрик приведены в таблице:

****



\clearpage